\documentclass[12pt]{article}

\usepackage[spanish]{babel}
 
\usepackage[margin=1.5cm,top=1cm]{geometry} 
\usepackage[utf8]{inputenc}
\usepackage[T1]{fontenc}
\usepackage[dvips]{graphicx}
\usepackage{xcolor}
\usepackage{times}
\usepackage{amsmath,amsthm,amssymb}
\usepackage{dsfont}
\usepackage{slashed}
\usepackage{mathtools}
\usepackage[shortlabels]{enumitem}
\usepackage{tikz}
\usetikzlibrary{calc}

\usepackage{amssymb}
\usepackage{float}
\usepackage{caption}
\usepackage{subcaption}
\usepackage{amsthm}
\usepackage{tensor}



\begin{document}
 
% --------------------------------------------------------------
%                         Start here
% --------------------------------------------------------------
 
\title{\Large{\textit{De Boltzmann al método de ecuación de Boltzmann en redes para simulación de flujos}} \\
Evidencia final \\ Laboratorio Avanzado AG-DIC 2024}
\date{\today}
\author{\\ José Pablo Cuevas Cázares }
\maketitle
\tableofcontents

\begin {section} {Introducción} 

A manera de preámbulo, la charla comienza con una introducción pictórica de la descripción de un fluido abordado desde " arriba a abajo" y de " abajo a arriba", ideada por Hasslacher durante su estancia en el Laboratorio Nacional de Los Álamos. En ella, Rechtman explica cómo empezar de arriba implica analizar el fluido como un conjunto infinitesimal de puntos (un continuo) sujeto a ciertas restricciones físicas (conservación de momento, conservación de energía etc.) y obtener las ecuaciones de Navier-Stokes. Por otro lado, empezar desde abajo implica asumir que el fluid se compone de partículas que obedecen la ecuación de Boltzmann, y a partir de ahí, aproximar para obtener la dinámica de fluidos.

 Brevemente, se expone después la historia de la dinámica de fluidos discreta, comenzando por las ecuaciones de Navier-Stokes, que son la base del campo de estudio, y continuando con los múltiples desarrollos por parte de Ulam, Chapman y Neumann  (entre otros) enfocados en discretizar las ecuaciones que rigen la dinámica de fluidos, desarrollando los autómatas celulares (intentos de simular sistemas vivos por medio de reglas simples implementadas computacionalmente) hasta llegar al método de redes de la ecuación de Boltzmann, más conocido simplemente como método de redes de Boltzmann.
 
 \end {section}
 \begin {section} {Ecuación de Boltzmann}
 Para introducir el método, se implementa la función de distribución de velocidades en el espacio fase $6N$ dimensional (siendo $N$ el número de partículas del fluido) definida como el número promedio de partículas en un elemento de volumen $d^3xd^3v$ alrededor de $\Vec{x}, \Vec{v}$ a un tiempo $t$. Matemáticamente, 

\begin {equation}
	f(\Vec{x},\Vec{v},t)d^3xd^3v
\end {equation}

Asumiendo independencia eventual (sin darse cuenta de ello) entre las funciones de colisión entre pares de partículas, conocida como la \textit{hipótesis de caos molecular}, Boltzmann obtuvo una integral de colisión para el sistema, es decir, una ecuación integro-diferencial no local que describía correctamente las colisiones del fluido, i.e., la ecuación de transporte de Boltzmann. 
Al ser una ecuación integro-diferencial, los métodos de solución en la época de Boltzmann eran prácticamente inexistentes, por lo que su genialidad radicó en la introducción del teorema H, con el cual, mediante el ansatz $\dfrac{dH}{dt}\leqq 0 $, siendo $H$
\begin {equation*}
H (t) = \int {d^{D}v\int {d^{D}x f (\Vec{x}, \Vec{v}, t)\log{\left(f (\Vec{x}, \Vec{v}, t)\right)}}}
\end {equation*}

A partir de ello, basándose y corrigiendo las ideas de Maxwell pudo obtener la famosa \textit{distribución de Maxwell-Boltzmann}, sin tener que resolver la integral de colisiones y asumiendo equilibrio termodinámico.

A lo largo de la misma línea de pensamiento, Boltzmann pudo obtener la entropía de un gas ideal, dando así la primer demostración en la historia de la segunda ley de la termodinámica a partir de un análisis mecánico-estadístico. Sin embargo, fue Loschmidt, amigo y contemporáneo de Boltzmann, el que corrigió el análisis anterior, ya que las leyes de la mecánica son inherentemente reversibles, mientras que la segunda ley es por naturaleza irreversible, lo que implica que una no puede ser obtenida a partir de principios de la otra. 
\end {section}

\begin {section} {Método de redes de la ecuación de Boltzmann} 

De manera análoga a la introducción a la mecánica de fluidos, la historia del método de redes es larga y complicada. Existen también distintos nombres acuñados para el algoritmo, difiriendo en los detalles acerca de la metodología de discretización de la ecuación de transporte de Boltzmann. 
\\
A partir de la ecuación de transporte de Boltzmann

\begin {align*}
\dfrac{df}{dt} \left(\Vec{x}, \Vec{v}, t\right) &= \left (\dfrac{\partial}{\partial t}+ \Vec{v}\cdot\nabla_{x} + \dfrac{\vec{F}}{m}\right)f \left(\Vec{x}, \Vec{v}, t\right)  \\
& =\int{d^{3}v_1\int{d\Omega \sigma (\Omega)|\Vec{v}_1-\Vec{v}|\left[f \left(\Vec{x}, \Vec{v}^{\prime}, t\right)f \left(\Vec{x}, \Vec{v}^{\prime}_{1}, t\right)-f \left(\Vec{x}, \Vec{v}^{\prime}_{1}, t\right)f \left(\Vec{x}, \Vec{v}_1, t\right)\right]}}
\end {align*}

y empleando la aproximación BGK (Bhatnagar-Gross-Krook), donde se reemplaza la integral de colisión por una relajación al equilibrio local de la función de distribución de Maxwell-Boltzmann $f^{eq}$ se sigue que

\begin {equation*}
\dfrac{df}{dt} \left(\Vec{x}, \Vec{v}, t\right) = \left (\dfrac{\partial}{\partial t}+ \Vec{v}\cdot\nabla_{x} + \dfrac{\vec{F}}{m}\right)f \left(\Vec{x}, \Vec{v}, t\right) = \dfrac{1}{\tau} \left[f^{eq} \left(\Vec{x}, \Vec{v}, t\right)-f \left(\Vec{x}, \Vec{v}, t\right)\right]
\end {equation*}

donde $\tau$ es el tiempo de relajación, parámetro que está influido por la viscosidad del fluido en cuestión, y 

\begin{equation*}
f^{eq} \left(\Vec{x}, \Vec{v}, t\right)  = n \left(\dfrac{m}{2\pi k T}\right)^{D/2}\exp{\left[-m\dfrac{(\Vec{v}-\Vec{u})^2}{2kT}\right]}
\end{equation*}
la función de distribución de equilibrio local de Maxwell-Boltzmann, con parámetros
$n = n \left(\Vec{x},t\right)$, $\Vec{u} = \Vec{u} \left(\Vec{x},t\right)$, $T = T \left(\Vec{x},t\right)$, número de densidad local, velocidad y temperatura respectivamente.
\\
Con ello, el método consiste en discretizar la ecuación anterior, que es dependiente de la red en la que se discretize la misma, de manera que la evolución de la función de distribución quede descrita por (en dos dimensiones) nueve nodos de velocidad dispuestos sobre una red cuadriculada cuya evolución depende de la expansión de la función exponencial a segundo orden en la velocidad ($\Vec{u}$), es decir, a número de Mach bajos. 

Entonces, algorítmicamente, se encuentra la evolución de la función de distribución promediando sobre los 9 nodos de la red, calculando los coeficientes relevantes y evolucionando en el tiempo de manera discreta. Junto con condiciones de frontera y condiciones iniciales, el método puede ser programado para proveer resultados iterativos del flujo del sistema que se desee estudiar. 
Sin embargo, como se señala en la discusión de la charla, tiene sus limitantes, ya que la introducción explícita de la temperatura conduce a problemas de discretización como consecuencia de las condiciones de equilibrio que debe cumplir el sistema en cada paso del tiempo. Otro de los problemas que surgen es el de garantizar que en cada espacio temporal se garantice la existencia de nueve funciones de distribución correspondientes a cada nodo, ya que si una de ellas se "pierde" en la evolución del sistema el método deja de funcionar correctamente. Así mismo, existe una "fuerza de cuerpo" externa, inherente al método, cuya forma varía según el sistema estudiado.

\begin {subsection} {Flujo de Poiseuille}
Uno de los ejemplos más "sencillos" a los cuáles se puede aplicar el método es el flujo laminar de Poiseuille, un flujo laminar en un tubo. La ecuación de Navier-Stokes que corresponde a este sistema implica la conservación del momento en una dirección arbitraria (en la dirección $x$, por ejemplo) es

\begin {equation*}
\dfrac{\partial u^{\prime}}{\partial t^{\prime}} = \nu \dfrac{\partial^2u^{\prime}}{\partial y^{\prime 2}} + \dfrac{F}{\rho}
\end {equation*}

donde $u^{\prime}$ es la componente horizontal de la velocidad, $t^{\prime}$ el tiempo, $y^{\prime}$ la posición vertical, $\nu$ la viscosidad cinemática, $\rho$ la densidad y $F$ el gradiente de presión, que se asume constante en todo el tubo y finalmente las cantidades primadas indican cantidades dimensionales. Se asumen también condiciones de frontera periódicas de frontera en las paredes del tubo. Al ser un flujo laminar, es posible encontrar una solución analítica a la EDP de Navier-Stokes por medio de series de Fourier, lo cual provee una manera de comparar la simulación del mismo sistema mediante redes de Boltzmann y la solución analítica. 
\\
A continuación, el autor procede a mostrar las gráficas comparativas entre el método de simulación y la solución analítica, donde se observa una similitud extraordinaria entre ambas soluciones, y además, para este caso, la simulación también es rápida compara con otros sistemas más complejos. Para el flujo de Poiseuille, la fuerza de cuerpo antes mencionada debe adquirir una forma especial, de tal forma que cambie la masa del fluido en cada iteración pero no la cantidad del momento del mismo, es decir, debe cumplir físicamente con la conservación del momento. 
\end {subsection}
Finalmente, la presentación concluye que existen muchas maneras diferentes de implementar el mismo método para la descripción del mismo sistema, pero que el método subyacente y la física sobre la cuál descansa es el mismo sin importar las vertientes que se desarrollen para la descripción de un sistema fluido en particular. 
\end {section}

\begin {section} {Conclusión}
En resumen, el método (o métodos) de redes en Boltzmann provee una alternativa útil a los algoritmos convencionales de dinámica de fluidos computacional (CFD), ya que mientras los métodos tradicionales se enfocan en la construcción de modelos que reproduzcan cantidades macroscópicas de un sistema fluido, como la presión, campo de velocidades, densidad etc., a través de la conservación de cantidades como el momento, masa y energía, el método de redes no obvia las propiedades moleculares del fluido, lo que inherentemente incrementa su precisión en la descripción de sistemas fluidos donde las interacciones moleculares son incrementalmente relevantes. Así mismo, al no lidiar directamente con las posiciones de las partículas de fluido consideradas en el sistema, si no que es la densidad del fluido la cual se modela a través de el promedio de la función de distribución en cada nodo de la red cuadriculada del método. Lo anterior entonces elimina la necesidad de rastrear la posición de cada partícula del fluido en cada iteración temporal, aunque ello puede llevar también a pérdidas de información sobre el transporte de calor en el mismo, ya que se asume una relajación (equilibrio termodinámico) a lo largo de la simulación.
\end {section}

\begin{thebibliography}{99}
\bibitem{video}
https://www.youtube.com/watch?v=U7TSAhBFoR4

\end {thebibliography}






\end{document}












