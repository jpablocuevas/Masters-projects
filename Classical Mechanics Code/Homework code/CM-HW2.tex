% --------------------------------------------------------------
% This is all preamble stuff that you don't have to worry about.
% Head down to where it says "Start here"
% --------------------------------------------------------------
 
\documentclass[12pt]{article}
 
\usepackage[margin=1.5cm,top=1cm]{geometry} 
\usepackage[utf8]{inputenc}
\usepackage[T1]{fontenc}
\usepackage[dvips]{graphicx}
\usepackage{xcolor}
\usepackage{times}
\usepackage{amsmath,amsthm,amssymb}
\usepackage{dsfont}
\usepackage{slashed}
\usepackage{mathtools}
\usepackage[shortlabels]{enumitem}
\usepackage{tikz}
\usetikzlibrary{calc}



\newenvironment{problem}[2][Problem]{\begin{trivlist}
\item[\hskip \labelsep {\bfseries #1}\hskip \labelsep {\bfseries #2.}]}{\end{trivlist}}

\begin{document}
 
% --------------------------------------------------------------
%                         Start here
% --------------------------------------------------------------
 
\title{Homework 2 Classical Mechanics\\Deadline August 30, 2023.}
\date{}
 
\maketitle

\begin{problem}{1} Suppose that a closed system of $N$ particles described by cartesian position vectors $\mathbf{r}_a$ is invariant under infinitesimal transformations
\begin{equation}
\delta\mathbf{r}_a=\epsilon \mathbf{r}_a,
\end{equation}
where $\epsilon$ is an ifinitesimal transformation parameter independent of time.
\begin{enumerate}[(a)]
\item Find the corresponding transformation for velocities $\delta\mathbf{v}_a$.
\item Show that, using the equations of motion the transformation of the Lagrangian
\begin{equation}
\delta L=\frac{\partial L}{\partial \mathbf{r}_a}\delta\mathbf{r}_a+\frac{\partial L}{\partial \mathbf{v}_a}\delta\mathbf{v}_a,
\end{equation}
can be written as a total derivative with respect to time.
\item Determine the conserved quantity associated to the symmetry $\delta\mathbf{r}_a=\epsilon \mathbf{r}_a$. Is this quantity conserved in the case of a free particle?
\end{enumerate}
\end{problem}


\begin{problem}{2} Consider a system described by the following Lagrangian
\begin{equation}
L=\frac{m}{2}(\dot{x}^2+\dot{y}^2+\dot{z}^2)-\frac{k}{2}(x^2+y^2),
\end{equation}
where $m$ and $k$ are positive constants.
\begin{enumerate}[(a)]
\item Write down the Euler-Lagrange Equations of Motion.
\item Show explicitly that the energy of the system $E$ is conserved, \emph{i.e.} $\frac{dE}{dt}=0$.
\item Calculate $\frac{dp_x}{dt}$, $\frac{dp_y}{dt}$ and $\frac{dp_z}{dt}$ on-shell and determine which components of the momentum are conserved.
\item Calculate $\frac{dL_x}{dt}$, $\frac{dL_y}{dt}$ and $\frac{dL_z}{dt}$ on-shell and determine which components of the angular momentum are conserved.\end{enumerate}
\end{problem}


\begin{problem}{3}
\begin{enumerate}[(a)]
\item Show that the system in Landau's Problem 4, page 12,  is described by the Lagrangian
\begin{equation}
L(\theta,\dot{\theta})=m_1 a^2(\dot{\theta}^2+\Omega^2\sin^2\theta)+2m_2 a^2 \dot{\theta}^2 \sin^2\theta+2(m_1+m_2)g a \cos\theta.
\end{equation}
\item Write down the corresponding equations of motion.
\item Find the energy of the system $E(\theta,\dot{\theta})$ and determine if it is conserved.
\item Show that $E\neq T+U$ and explain why the equality does not hold.
\item Find the generalized momentum and generalized force.
\item Discuss under what conditions is the generalized momentum constant in this system.
\end{enumerate}
\end{problem}
\newpage

\begin{problem}{4}
Two particles of masses $m_1$ and $m_2$ are connected by an inextensible string of length $l$. The first particle is confined to move on the $xy$ plane (the surface of a table), while the second moves only along the $z$ axis (the string passes through a hole at the origin). A constant gravitational acceleration, $g$ acts in the negative $z$ direction.
\begin{enumerate}[(a)]
\item Write the Lagrangian $L(r,\phi,\dot{r},\dot{\phi})$, where $r$ and $\phi$ are the cylindrical coordinates of the first particle.
\item Find the energy of the system $E$ and the angular momentum about the $z$ axis $L_{z}$, and explain why we expect them to be conserved. Are there any other integrals of motion?
\item Under what condition does $m_1$ undergo circular motion? What is the energy associated with this orbit?
\item Show that the motion can be expressed as $\frac{1}{2}(m_1+m_2)\dot{r}^2=E-V_{\text{eff}}(r)$ .
%\item Graph $V_{\text{eff}}(r)$ taking  the values $m_1=1\,\mathrm{kg}$, $m_2=2\,\mathrm{kg}$, $l=10\,\mathrm{m}$ and $g=9.8\,\mathrm{m/s^2}$ for the cases $L_z=1,\,10,\,20,\,30,\,\mathrm{J\,s}$.
%\item for $E=-50 \,\mathrm{J}$ determine the minimum and maximum values of $r$, taking $m_1=1\,\mathrm{kg}$, $m_2=2\,\mathrm{kg}$, $l=10\,\mathrm{m}$ and $g=9.8\,\mathrm{m/s^2}$ for the cases $L_z=1,\,10,\,20,\,30,\,\mathrm{J\,s}$
\end{enumerate}

\begin{center}
\begin{tikzpicture}[scale=1.5]
\draw [dashed] (0,4) circle (2cm and 0.4cm);
\draw[->,dashed] (0,0) -- (0,5); 
\draw[->,dashed] (2,5) -- (-2,3); 
\draw[-,thick,red] (0,4) -- (0,1); 
\node at (0,1)[circle,fill,inner sep=3pt,blue] {};
\node at (0.5,1)[]{$m_2$};
\draw[-,thick,red] (0,4) -- (1.5,3.75); 
\node at (1.5,3.75)[circle,fill,inner sep=3pt,blue] {};
\node at (2,3.5)[]{$m_1$};
\draw[green,thick] (-0.5,3.75) arc (-100:-75:3.5);
\node at (0.5,3.4)[]{$\phi$};
\end{tikzpicture}
\end{center}
\end{problem}

 
\end{document}