% --------------------------------------------------------------
% This is all preamble stuff that you don't have to worry about.
% Head down to where it says "Start here"
% --------------------------------------------------------------
 
\documentclass[12pt]{article}
 
\usepackage[margin=1.5cm,top=1cm]{geometry} 
\usepackage[utf8]{inputenc}
\usepackage[T1]{fontenc}
\usepackage[dvips]{graphicx}
\usepackage{xcolor}
\usepackage{times}
\usepackage{amsmath,amsthm,amssymb}
\usepackage{dsfont}
\usepackage{slashed}
\usepackage{mathtools}
\usepackage[shortlabels]{enumitem}
\usepackage{tikz}
\usetikzlibrary{calc}



\newenvironment{problem}[2][Problem]{\begin{trivlist}
\item[\hskip \labelsep {\bfseries #1}\hskip \labelsep {\bfseries #2.}]}{\end{trivlist}}

\begin{document}
 
% --------------------------------------------------------------
%                         Start here
% --------------------------------------------------------------
 
\title{Homework 3 Classical Mechanics\\Deadline September 18, 2023.}
\date{}
 
\maketitle

\begin{problem}{1} 
Show that a one-dimensional particle subject to a potential $U(x) =k x^{2(n+1)}$, where $n$ is an integer (with $n\neq-1$) and $k$ a constant, will oscillate with a period proportional to $A^{-n}$ where $A$ is the amplitude.
\end{problem}

\begin{problem}{2}
A particle of mass $m$ moves along the $x$ axis under the influence of
the potential
\begin{equation}
U(x)=U_0 x^2 e^{-a x^2},
\end{equation}
where $U_0$ and $a$ are positive constants. 
\begin{enumerate} [(a)]
\item Find the equilibrium points of the motion, 
\item Plot the potential energy.
\item Draw the phase portrait of the system. Indicate the relation between the energy and the geometry of the phase-space
orbits, showing the different phases of mechanical motion.
\end{enumerate}
\end{problem}


\begin{problem}{3}
A point particle of unit mass is moving along a line under the action of a conservative force field with potential energy
\begin{equation}
U(x)=\frac{\kappa}{2} x^2+\frac{\lambda}{4}x^4,
\end{equation}
where $\kappa$ and $\lambda$ are two given real parameters.
\begin{enumerate}[(a)]
\item Determine  all the positions of stable equilibrium for all the values of the parameters $\kappa$ and $\lambda$.
\item Consider the motion, with initial conditions $x(0) = 0$, $\dot{x}(0) = 1$. For which values of $\kappa$ and $\lambda$ is the motion periodic? 
\item For the same initial conditions $x(0) = 0$, $\dot{x}(0) = 1$, determine the range of values of $\kappa$ and $\lambda$ where the particle goes to infinity in finite time.
\item Determine all the periodic motions and the corresponding time period.
\item Draw the phase portrait of the system in the following cases: $\kappa>0$ and $\lambda>0$,
\item $\kappa<0$ and $\lambda<0$,
\item $\kappa<0$ and $\lambda>0$,
\item $\kappa>0$ and $\lambda<0$.
\end{enumerate}
\end{problem}
\newpage


 
\end{document}