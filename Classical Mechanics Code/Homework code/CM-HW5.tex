% --------------------------------------------------------------
% This is all preamble stuff that you don't have to worry about.
% Head down to where it says "Start here"
% --------------------------------------------------------------
 
\documentclass[12pt]{article}
 
\usepackage[margin=1.5cm,top=1cm]{geometry} 
\usepackage[utf8]{inputenc}
\usepackage[T1]{fontenc}
\usepackage[dvips]{graphicx}
\usepackage{xcolor}
\usepackage{times}
\usepackage{amsmath,amsthm,amssymb}
\usepackage{dsfont}
\usepackage{slashed}
\usepackage{mathtools}
\usepackage[shortlabels]{enumitem}
\usepackage{tikz}
\usetikzlibrary{calc}



\newenvironment{problem}[2][Problem]{\begin{trivlist}
\item[\hskip \labelsep {\bfseries #1}\hskip \labelsep {\bfseries #2.}]}{\end{trivlist}}

\begin{document}
 
% --------------------------------------------------------------
%                         Start here
% --------------------------------------------------------------
 
\title{Homework 5 Classical Mechanics\\Deadline November 6, 2023.}
\date{}
 
\maketitle


\begin{problem}{1}
Determine whether the transformation
 \begin{align}
&Q_1=\beta q_1\cos\alpha+ p_2\sin\alpha,&\quad&
&Q_2=\beta q_2\cos\alpha + p_1\sin\alpha,\\
&P_1=-q_2\sin\alpha+\frac{p_1}{\beta}\cos \alpha,&\quad&
&P_2=-q_1\sin\alpha +\frac{p_2}{\beta}\cos\alpha ,
\end{align}
is canonical.
\end{problem}


\begin{problem}{2}
Use the Canonical transformation generated by
\begin{equation}
F(q,Q)= \frac{m\omega}{2}q^2 \cot Q.
\end{equation}
 to solve the harmonic oscillator described by the Hamiltonian
\begin{equation}
H= \frac{p^2}{2m}+\frac{m\omega^2}{2} q^2.
\end{equation}
\end{problem}

\begin{problem}{3}
(a) Prove that the transformation
\begin{equation}
Q =\frac{p}{\tan q},\qquad
P = \log\left(\frac{\sin q}{p}\right),
\end{equation}
is canonical. (b) Find the generating function $F(q, Q)$ for this transformation.
\end{problem}

\begin{problem}{4}
Given the Hamiltonian
\begin{equation}
H=\alpha p_2^2 + \beta p_1^2 + p_1 q_1 - p_2 q_2,
\end{equation}
Show that the three functions
\begin{equation}
f_1=(\beta p_1 + q_1)/p_2,\qquad
f_2=p_1 p_2,\qquad
f_3=p_2 \exp(-t),
\end{equation}
are constants of motion.  Are they functionally independent? Do there exist other
independent constants of motion? 
\end{problem}

\begin{problem}{5}
For a system described by the Hamiltonian
\begin{equation}
H=\alpha |\mathbf{p}|^n + \beta  |\mathbf{r}|^{-n},
\end{equation}
where $\mathbf{p}$ is the vector of the momenta conjugate to the cartesian coordinates $\mathbf{r}$ and $n$ is a constant. Show that there is a conserved quantity given by
\begin{equation}
D=\frac{\mathbf{p}\cdot\mathbf{r}}{n}-Ht.
\end{equation}
Are there more independent conserved quantities? 
\end{problem}

\begin{problem}{6}
For the two-dimensional isotropic harmonic oscillator 
\begin{equation}
H=\frac{p_1^2}{2m}+\frac{p_2^2}{2m} + \frac{m \omega^2}{2}q_1^2 +\frac{m \omega^2}{2}q_2^2,
\end{equation}
(a) verify that the functions 
\begin{equation}
S_1=\frac{p_1p_2}{2 m \omega}+\frac{m\omega q_1 q_2}{2} ,\qquad S_2=\frac{1}{4 m\omega}\left[p_2^2 - p_1^2+m^2\omega^2(q_2^2 - q_1^2) \right],
\end{equation}
are conserved quantities. (b) Find another constant of motion from the Poisson theorem $S_3\equiv\{S_1,S_2\}$ and identify its physical meaning. (c) Determine the Poisson bracket algebra $\{S_i,S_j\}$, $i,j=1,\dots,3$. (d) Show that the relation
\begin{equation}
S_1^2+S_2^2+S_3^2=\frac{H^2}{4\omega^2},
\end{equation} 
holds.
\end{problem}

\begin{problem}{7}
In spherical coordinates the Hamiltonian for a particular system is given by
\begin{equation}
H=\frac{1}{2m}\left(p_r^2+\frac{p_\theta^2}{r^2}+\frac{p_\phi^2}{r^2\sin^2\theta}\right)+a(r)+\frac{b(\theta)}{r^2},
\end{equation}
where $a(r)$ and $b(\theta)$ are arbitrary functions.
(a) Write the Hamilton–Jacobi equation and separate variables. (b) Integrate to find the action $S$.
\end{problem}

 
\end{document}